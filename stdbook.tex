\documentclass[a4paper]{report}
%\usepackage[inline]{asymptote}
\usepackage{asymptote}
%\usepackage{csvsimple}
%\usepackage{numprint}
\usepackage{tabularx}
\usepackage{colortbl}
%\usepackage{spreadtab}
%\usepackage{expl3}
\ExplSyntaxOn
% make an internal function available to the user
\cs_set_eq:NN \fpeval \fp_eval:n
\ExplSyntaxOff
\def\asydir{asy}
%%%%%%%%%%%%%%%%%%%%%%%%%%%%%%%%%%%%%%%%
%latexmk stdbokk.tex -pvc
%live-server www.html in folder//npm install -g live-server ///
%while true; do inotifywait -e CLOSE_WRITE www.asy; asy -f html  www.asy; done; ///
%%%%%%%%%%%%%%%%%%%%%%%%%%%%%%%%%%%%%%%%
%https://github.com/fissart/stdbook/raw/master/stdbook.pdf
\usepackage[spanish]{babel}
\usepackage[centertags]{amsmath}
\usepackage{amsfonts}

\usepackage{graphicx}\usepackage{longtable}
\usepackage{booktabs}
%\usepackage{ulem}
%\usepackage{textcomp}
%\usepackage{showframe}
%\usepackage[utf8]{inputenc}
%\usepackage{hyperref}
%%%%%%%%%%%%%%%%%%%%%%%%%%%%%%%%%%%%%%%%
\usepackage[]{apacite}%apaciteclassic, nosectionbib, tocbib
\usepackage{usebib}
\bibinput{bb}
\usepackage{makeidx}
\makeindex
%%%%%%%%%%%%%%%%%%%%%%%%%%%%%%%%%%%%%%%%
\newtheorem{rem}{Comentario}[chapter]
\newtheorem{thm}{Teorema}[chapter]
\newtheorem{defn}[thm]{Definición}
\newtheorem{prop}{Proposición}[thm]
\newtheorem{lem}{Lema}[thm]
\newtheorem{cor}{Corolario}[thm]
\newtheorem{ill}{Ilustración}[thm]
%%%%%%%%%%%%%%%%%%%%%%%%%%%%%%%%%%%%%%%%

\newcommand{\qw}{\phi}
\newcommand{\Real}{\mathbb R}
\newcommand{\pa}[1]{\left(#1\right)}
%%%%%%%%%%%%%%%%%%%%%%%%%%%%%%%%%%%%%%%%
%%%%%%%%%%%%%%%%%%%%%%%%%%%%%%%%%%%%%%%%
%%%%%%%%%%%%%%%%%%%%%%%%%%%%%%%%%%%%%%%%


\begin{document}
%%%%%%%%%%%%%%%%%%%%%%%%%%%%%%%%%%%%%%%%
\begin{asydef}
	settings.prc=false;
	defaultpen(fontsize(11 pt));
	defaultpen(linewidth(0.7pt));
	settings.render=2;
\end{asydef}
%%%%%%%%%%%%%%%%%%%%%%%%%%%%%%%%%%%%%%%%
\thispagestyle{empty}

{
	\centering
	\vspace{3cm}
	\bf{\huge STATISTIC}\\
	\bf{\large STATISTIC}\\
	\vspace{0.5cm}
	\bf{RICARDO}\\
	\vspace{0.5cm}

	\begin{asy}
		import three;
		import palette;
		import graph3;
		size(11cm,0);
		currentprojection=perspective(-30,-30,30,up=Z);
		surface s;
		for(int i = 0; i < 10; ++i) {
				for(int j = 0; j < 10; ++j) {
						s.append(shift(i,j,0)*scale(1,1,i+j)*unitcube);
					}
			}
		s.colors(palette(s.map(zpart),Rainbow()));
		draw(s,meshpen=orange+thick(),nolight,render(merge=true));
		xaxis3("$x$",Bounds,InTicks(endlabel=false,Label,2,2));
		yaxis3(YZ()*"$y$",Bounds,InTicks(beginlabel=false,Label,2,2));
		zaxis3(XZ()*"$z$",Bounds,InTicks);
	\end{asy}
}

\vfill

\noindent
%ps://asy.marris.fr/
Departamento de matemática y física, FIMGC-USNCH\\
\emph{E-mail}: \texttt{ricardomallqui6@gmail.com}\\
URL: \textsf{www.fractales.com}

\newpage
%%%%%%%%%%%%%%%%%%%%%%%%%%%%%%%%%%%%%%%%

{
	\thispagestyle{empty}
	\noindent\bf{Estadística en artistas}\\
	\bf{Ricardo Michel Mallqui Baños}\\
	\vspace{3cm}

	\noindent Un libro de estadística, basado en código Asymptote y LaTeX.\\

	\noindent Bibliografia.\\
	\noindent Incluye Indice.\\
	1. Estadística descriptiva 2. Estadística Inferencial\\
	\vfill
	\noindent
	\begin{asy}
		import graph;
		size(0,150);

		int a=-1, b=1;

		real f(real x) {return x^3-x+2;}
		real g(real x) {return x^2;}

		draw(graph(f,a,b,operator ..),red);
		draw(graph(g,a,b,operator ..),blue);

		xaxis();

		int n=5;

		real width=(b-a)/(real) n;
		for(int i=0; i <= n; ++i) {
				real x=a+width*i;
				draw((x,g(x))--(x,f(x)));
			}

		labelx("$a$",a);
		labelx("$b$",b);
		draw((a,0)--(a,g(a)),dashed);
		draw((b,0)--(b,g(b)),dashed);

		real m=a+0.73*(b-a);
		arrow("$f(x)$",(m,f(m)),N,red);
		arrow("$g(x)$",(m,g(m)),E,0.8cm,blue);

		int j=2;
		real xi=b-j*width;
		real xp=xi+width;
		real xm=0.5*(xi+xp);
		pair dot=(xm,0.5*(f(xm)+g(xm)));
		dot(dot,darkgreen+4.0);
		arrow("$\left(x,\frac{f(x)+g(x)}{2}\right)$",dot,NE,2cm,darkgreen);
	\end{asy}

	%\noindent \texttt{\textregistered\;\textcopyright\; 2023 pa-esfa, Inc. UNSCH, Huamanga}\\
	\noindent %\texttt{\textregistered\;\textcopyright}\\
	\texttt{Todos los derechos reservados. Ninguna parte de esto
		libro puede ser reproducido en cualquier forma,
		o por cualquier medio, sin permiso
		por escrito del editor.}\\
	Departamento de matemática y física, FIMGC-USNCH\\
	\emph{E-mail}: \texttt{ricardomallqui6@gmail.com}\\
	URL: \textsf{www.fractales.com}

}
%%%%%%%%%%%%%%%%%%%%%%%%%%%%%%%%%%%%%%%%
\newpage
\renewcommand\listfigurename{Índice general}
\pagenumbering{roman}
\setcounter{page}{1}
\addcontentsline{toc}{chapter}{Índice general}
\tableofcontents

\renewcommand\listfigurename{Lista de figuras}
\addcontentsline{toc}{chapter}{Lista de figuras}
\listoffigures

\renewcommand\listtablename{Lista de tablas}
\addcontentsline{toc}{chapter}{Lista de tablas}
\listoftables
\newpage

\clearpage
%%%%%%%%%%%%%%%%%%%%%%%%%%%%%%%%%%%%%%%%

\chapter*{Presentación}

Este libro se elaboró

\addcontentsline{toc}{chapter}{Presentación}

\underline{\underline{Double underlined text}}
{Double underlined text}

\textsl{\underline{Slanted underlined}}
\textsc{\underline{Small caps underlined}}








%%%%%%%%%%%%%%%%%%%%%%%%%%%%%%%%%%%%%%%%111
%%%%%%%%%%%%%%%%%%%%%%%%%%%%%%%%%%%%%%%%
%%%%%%%%%%%%%%%%%%%%%%%%%%%%%%%%%%%%%%%%
%%%%%%%%%%%%%%%%%%%%%%%%%%%%%%%%%%%%%%%%
%%%%%%%%%%%%%%%%%%%%%%%%%%%%%%%%%%%%%%%%
%%%%%%%%%%%%%%%%%%%%%%%%%%%%%%%%%%%%%%%%
%%%%%%%%%%%%%%%%%%%%%%%%%%%%%%%%%%%%%%%%
%%%%%%%%%%%%%%%%%%%%%%%%%%%%%%%%%%%%%%%%
%%%%%%%%%%%%%%%%%%%%%%%%%%%%%%%%%%%%%%%%
%%%%%%%%%%%%%%%%%%%%%%%%%%%%%%%%%%%%%%%%
\part{Estadística descriptiva}
\pagenumbering{arabic}
\setcounter{page}{1}

\chapter{Preliminares}

\section{Conceptos básicos}

\begin{defn}[Poblacion]
	A statistical population is a data set (usually large, sometimes conceptual) that is our target of interest.
\end{defn}



\begin{defn}[Poblacion]
	A statistical population is a data set (usually large, sometimes conceptual) that is our target of interest.
\end{defn}
\begin{defn}[Poblacion]
	A statistical population is a data set (usually large, sometimes conceptual) that is our target of interest.
\end{defn}
\begin{defn}[Poblacion]
	A statistical population is a data set (usually large, sometimes conceptual) that is our target of interest.
\end{defn}
\begin{defn}[Poblacion]
	A statistical population is a data set (usually large, sometimes conceptual) that is our target of interest.
\end{defn}
\begin{thm}
	wwwwwwwwwwwwwwwww
\end{thm}


\subsection{Variables estadísticas}

\begin{defn}[Variables cuantitativas]
	Una variable estadística es una característica que puede fluctuar y cuya variación es susceptible de adoptar diferentes valores, los cuales pueden medirse u observarse. Las variables adquieren valor cuando se relacionan con otras variables, es decir, si forman parte de una hipótesis o de una teoría. Existen dos clases de variables: Cualitativas y cuantitativas.
	\begin{enumerate}
		\item  Cualitativas. Son aquellas variables que están propensos a ser nominadas textualmente.
		      \begin{enumerate}
			      \item  Nominales. Son características que simplemente nominan y están propensos a ser jerarquizados u ordenados tales como: El estado   civil (soltero, casado, divorciado, viudo), Religión (católica, evangélico, judío, etc).
			      \item  Ordinales. Son características que que si están propensos a ser jerarquizados tales como: Nivel de instrucción (inicial, primaria, secundaria, superior).
		      \end{enumerate}
		\item  Cuantitativas. Son aquellas variables que están propensos a ser medidas mediante números ya sean números enteros o reales.
		      \begin{enumerate}
			      \item  Discretas. Aquellas que solo son medidos mediante números enteros por ejemplo: Número de hijos y número de habitaciones.
			      \item  Continuas. Aquellas que solo son medidos mediante números reales es decir este incluye a los números racionales e irracionales. Estatura, volumen, peso.
		      \end{enumerate}
	\end{enumerate}
\end{defn}


\section{Organización de datos en tablas de frecuencias}
\subsection{Distribución de frecuencias}
El uso de tablas de distribución de frecuencias y gráficas como un medio para presentar la información
de un conjunto de datos de forma resumida. En grados anteriores ya se ha trabajado con gráficas para
variables cuantitativas discretas, por lo que esta será la primera vez que el estudiante trabajará con
gráficas que son adecuadas para presentar información de variables cuantitativas continuas.

\begin{defn}
La tabulación es un proceso en el cual los datos son ordenados en grupos llamados clases para un análisis más eficaz de estos, los datos podrían estar clasificados mediante una variable cualitativa o cuantitativa en el caso de las variables cualitativas $Y_i$, se considera la siguiente Tabla \ref{tab:www}
\end{defn}


En el caso de variables cuantitativas ademas si los datos son muy variados, que para se clasificados adecuadamente, necesitan generarse particiones de longitudes semejantes entonces se utiliza el siguiente proceso; el número de las particiones $r$ se consideran de acuerdo a tres criterios.
\begin{enumerate}
	\item Criterio del investigador $r$ no puede ser más de 20 ni menos de 5
	\item $r=\sqrt{n}$ donde $n$ es el número de datos
	\item La regla de Starges que consiste en considerar la fórmula $r=3.322\cdot\log_{10} n$ Una vez establecido el número de particiones se procede a generar los límites laterales de cada una de las particiones, sea $L$ la longitud de todo el conjunto es decir $L=x_{\text{max}}-x_{\text{min}}$ entonces la longitud de las particiones o amplitud interválica se obtiene con $l=\frac{L}{r}$
\end{enumerate}



\begin{longtable}{>{\color{blue}}ccc>{\color{blue}}c>{\color{yellow}}cccccccccc}
	\caption{Combinaciones de los tres segmentos de la seccion aurea.}
	\label{tab:www}                                                                             \\
	\toprule
	$Y_i$ & $f_i$ & $F_i$ & $F_i*$ & $h_i$ & $H_i$ & $H_i$  & $h_i\%$ & $H_i\%$ & $H_i*\%$    \\
	\midrule
	\endfirsthead
	\multicolumn{8}{c}{{\bfseries \tablename\ \thetable{} -- continua de la página anterior}} \\
	\toprule
	$Y_i$ & $f_i$ & $F_i$ & $F_i*$ & $h_i$ & $H_i$ & $H_i$  & $h_i\%$ & $H_i\%$ & $H_i*\%$    \\
	\endhead
	\midrule
	\multicolumn{8}{c}{{Continúa en la proxima página}}                                       \\ \midrule
	\endfoot
	\bottomrule
	\endlastfoot
	2     & 1     & 1.00  & 20.00  & 0.05  & 0.05  & 0.0025 & 0.25    & 0.25    & 0.2500      \\
	3     & 2     & 3.00  & 19.00  & 0.10  & 0.15  & 0.0050 & 0.50    & 0.50    & 0.7500      \\
	4     & 5     & 8.00  & 17.00  & 0.25  & 0.40  & 0.0125 & 1.25    & 1.25    & 2.0000      \\
	5     & 7     & 15.00 & 12.00  & 0.35  & 0.75  & 0.0175 & 1.75    & 1.75    & 3.7500      \\
	6     & 4     & 19.00 & 5.00   & 0.20  & 0.95  & 0.0100 & 1.00    & 1.00    & 4.7500      \\
	7     & 1     & 20.00 & 1.00   & 0.05  & 1.00  & 0.0025 & 0.25    & 0.25    & 5.0000      \\
	      & 20    &       &        &       &       &        & 5       &         &             \\
\end{longtable}


Tenga en cuenta que $n$ es el número de datos, es decir $n=f_1+f_2+\ldots+f_r=\sum_{i=1}^r$ donde $f_i$ es número de datos en la partición $X_i$, una de las $r$ particiones del conjunto total de datos.

\begin{enumerate}
	\item  Las frecuencias absolutas\index{frecuencias absolutas} $f_i$ indican el número de datos con la característica $X_i$.
	\item   Las frecuencias absolutas acumuladas menor que\index{frecuencias absolutas acumuladas menor que} $F_i$ obedecen a la fórmula $$F_m=f_1+f_2+\ldots+f_m=\sum_{i=1}^mf_i$$
	\item   Las frecuencias absolutas acumuladas mayor que $F_i^*$ obedecen a la fórmula
	      $$
		      \begin{aligned}
			      F_m^* & =f_m+f_{m+1}+\ldots+f_r                  \\
			            & =\sum_{i=m}^rf_i                         \\
			            & =n-\sum_{i=1}^{m-1}f_i                   \\
			            & =n-\left(f_1+f_{2}+\ldots+f_{m-1}\right)
		      \end{aligned}
	      $$
	\item   Las frecuencias absolutas relativas\index{frecuencias absolutas relativas} obedecen a la fórmula $$h_m=\frac{f_m}{n}$$
	\item   Las frecuencias absolutas relativas menor que\index{frecuencias absolutas relativas  menor que} obedecen a la fórmula $$H_m=\frac{f_m}{n}$$
	\item   Las frecuencias absolutas relativas mayor que obedecen a la fórmula $$H_m^*=\frac{F_m}{n}$$
	\item   Las frecuencias absolutas relativas porcentuales obedecen a la fórmula $h_i\%=100\cdot h_i$
	\item  Las frecuencias absolutas relativas menor que porcentuales obedecen a la fórmula $H_i\%=100\cdot H_i$
	\item   Las frecuencias absolutas relativas mayor que porcentuales obedecen a la fórmula $H_i^*\%=100\cdot H_i^*$
	\item  $Y_i$ marca de clase o punto medio de la clase $i$
\end{enumerate}


\subsection{Distribución de frecuencias continuas}


\section{Gráficos estadísticos}

\subsection{Histograma de frecuencias}

\begin{figure}[!ht]
	\centering
	\begin{asy}
		import stats;
		import graph;
		size(12cm,6cm,false);
		real[] tabxi={1,5,9,10,12,15,20,28,35,38};
		real[] tabni={2,5,6,7, 8, 6, 3, 2, 1};
		real[] tabhi;
		real[] tabwi;
		for(int i=0; i < tabni.length; ++i){
		tabhi[i]=tabni[i]/(tabxi[i+1]-tabxi[i]);
		tabwi[i]=(tabxi[i+1]+tabxi[i])/2;
		draw((tabwi[i],tabhi[i])--(i==0?2*tabxi[0]-tabwi[i]:tabwi[i-1],i==0?0:tabhi[i-1]),orange);
		label(format( "%.1f",tabni[i]/sum(tabni)*100)+"\%",(tabwi[i],tabhi[i]),N );
		dot((tabwi[i],tabhi[i])^^(2*tabxi[0]-tabwi[0],0));
		}
		draw(((3*tabxi[tabwi.length]-tabxi[tabwi.length-1])/2,0)--(tabwi[tabwi.length-1],tabhi[tabwi.length-1]),orange);        
		dot(((3*tabxi[tabwi.length]-tabxi[tabwi.length-1])/2,0));
		write((tabxi[tabwi.length]-tabxi[tabwi.length-1])/2);
		histogram(tabxi,tabhi,low=0,bars=true,blue+opacity(0.5));
		xaxis("wwwwwwww",Bottom, RightTicks(Step=3,step=1),above=true);
		//shipout(bbox(8mm,orange));
	\end{asy}
	\caption{Histograma de frecuencias}
\end{figure}



\subsection{Gráficos circulares}

\begin{figure}[!ht]
	\centering
	\begin{asy}
		size(5cm);
		string[] www={"www","$\sum_2^3x_i$","\int_1^3fxdx","w","www"};
		real[] w1={20,30,10,10,50};
		real[] w2={0.1,0,0.2,0,0};
		real[] t1;
		real[] t2;
		real[] t3;
		int n=w1.length;
		t2[0]=0;
		for(int i=0; i < n; ++i ){
		t1[i]=w1[i]*360/sum(w1);
		t2[i+1]=t2[i]+t1[i];
		t3[i]=t2[i]+t1[i]/2;
		path sector=(0,0)--arc((0,0),2,t2[i],t2[i+1])--cycle;
		transform tt1=shift(w2[i]*dir(t3[i]));
		transform tt2=shift((2.7+w2[i])*dir(t3[i]));
		fill(tt1*sector, i/n*yellow+(1-i/n)*blue);
		label(www[i],tt2*(0,0),blue);
		}
	\end{asy}
	\caption{Circular}
\end{figure}


\subsection{Diagrama de barras}

\begin{figure}[!ht]
	\centering
	\begin{asy}
		size(7.5cm,0);
		pen dashed=linetype(new real[] {5,3});
		void bargraph(real X, real Y,
		real ymin, real ymax, real ystep,
		real tickwidth, string yformat,
		Label LX, Label LY, Label[] LLX,
		real[] height,
		pen p=nullpen){
		draw((0,0)--(0,Y),EndArrow);
		draw((0,0)--(X,0),EndArrow);
		label(LX,(X,0),plain.SE);
		label(LY,(0,Y),plain.N);//fontsize(9)
		real yscale=Y/(ymax+ystep);
		for(real y=ymin; y<ymax; y+=ystep) {
				draw((-tickwidth,yscale*y)--(0,yscale*y));
				label(format(yformat,y),(-tickwidth,yscale*y),plain.W);
			}
		int n=LLX.length;
		real xscale=X/(2*n+2);
		for(int i=0;i<n;++i) {
		real x=xscale*(2*i+1);
		path P=(x,0)--(x,height[i]*yscale)--(x+xscale,height[i]*yscale)--(x+xscale,0)--cycle;
		fill(P,p);
		draw(P);
		label(LLX[i],(x+xscale/2),plain.S);
		}
		for(int i=0;i<n;++i)
		draw((0,height[i]*yscale)--(X,height[i]*yscale),dashed);
		}
		string yf="%#.1f";
		Label[] LX={"wwwwww","www","www1","www2","www3","www"};
		for(int i=0;i<LX.length;++i) LX[i]=rotate(45)*LX[i];
		real[] H={3,5.3,12.9,21.3,9.8,4.3};

		bargraph(X=50,Y=70,
		ymin=2,ymax=23,ystep=5,
		tickwidth=2,
		yf,
		"$www$","$\theta_3$",
		LX,H,
		orange+white);
	\end{asy}
	\caption{Diagrama de barras}
\end{figure}



\begin{figure}[!ht]
	\centering
	\begin{asy}
		size(7cm,5cm,false);
		pair A=(8,37),B=(10,57.5);
		real yM=50;
		real xM=(yM-A.y)*(B.x-A.x)/(B.y-A.y)+A.x;
		real dx=.2(B.x-A.x), dy=.2(B.y-A.y);
		draw((A.x-dx,A.y-2dy)--(B.x+dx,A.y-2dy),.7bp+black);
		draw((A.x-2dx,A.y-dy)--(A.x-2dx,B.y+dy),.7bp+black);

		draw(A--B,1.2bp+black);
		dot("$A$",A,SW,blue);
		dot("$B$",B,NE,blue);
		dot("$M$",(xM,yM),SE,red);

		draw((A.x-2dx,A.y)--A--(A.x,A.y-2dy),dashed+.5bp+black);
		draw((A.x-2dx,B.y)--B--(B.x,A.y-2dy),dashed+.5bp+black);
		draw((A.x-2dx,yM)--(xM,yM)--(xM,A.y-2dy),dashed+.5bp+black);
		label(format("$%f$",A.x),(A.x,A.y-2dy),S);
			label(format("$%f$",A.y),(A.x-2dx,A.y),W);
		label(format("$%f$",B.x),(B.x,A.y-2dy),S);
			label(format("$%f$",B.y),(A.x-2dx,B.y),W);
		label("$x_M$?",(xM,A.y-2dy),S,red);
		label(format("$%f$",yM),(A.x-2dx,yM),W);
	\end{asy}
	\caption{Meadiana}
\end{figure}



\chapter{Medidas estadísticas de variables cuantitativas}

\section{Medidas de tendencia central}
Son aquellas medidas que buscan un dato representivo central de un conjunto de datos tales como la media, la moda y la mediana.

\begin{defn}[Datos agrupados y  no agrupados]
	La principal diferencia entre los datos agrupados y los no agrupados es que los agrupados están clasificados según un criterio y los no agrupados se encuentran en el mismo formato que cuando se \begin{rem}[]

	\end{rem}.
\end{defn}


\begin{longtable}{>{\color{blue}}ccc>{\color{blue}}cc>{\color{blue}}c}
	\caption{Combinaciones de los tres segmentos de la seccion aurea.}
	\label{tab:w}                                                                             \\
	\toprule
	Clase           & $Y_i$    & $f_i$    & $F_i$    & $\ldots$ & $H_i^*\%$                   \\
	\midrule
	\endfirsthead
	\multicolumn{5}{c}{{\bfseries \tablename\ \thetable{} -- continua de la página anterior}} \\
	\toprule
	Clase           & $Y_i$    & $f_i$    & $F_i$    & $\ldots$ & $H_i^*\%$                   \\
	\endhead
	\midrule
	\multicolumn{5}{c}{{Continúa en la proxima página}}                                       \\ \midrule
	\endfoot
	\bottomrule
	\endlastfoot
	$[y_1,y_2)$     & $y_1$    & $f_1$    & $\ldots$ & $\ldots$ & $H_1^*\%$                   \\
	$[y_2,y_3)$     & $y_2$    & $f_2$    & $\ldots$ & $\ldots$ & $H_1^*\%$                   \\
	$[y_3,y_4)$     & $y_3$    & $f_3$    & $\ldots$ & $\ldots$ & $H_1^*\%$                   \\
	$\vdots$        & $\vdots$ & $\vdots$ & $\ldots$ & $\ldots$ & $\vdots$                    \\
	$[y_{r-1},y_r]$ & $y_r$    & $f_r$    & $\ldots$ & $\ldots$ & $H_1^*\%$                   \\
\end{longtable}

\subsection{Media}


\subsection{Mediana}
\subsection{Moda}


\section{Medidas de dispersión}
\section{Medidas de variacion}



\chapter{Variables estadísticas bidimensionales} 

When we put (vertically) large expressions inside of parentheses (or brackets, or curly braces, etc.), the parentheses don't resize to fit the expression and instead remain relatively small. For instance, $$f(x) = \pi(\frac{\sqrt{x}}{x-1})$$ comes out as

When we put (vertically) large expressions inside of parentheses (or brackets, or curly braces, etc.), the parentheses don't resize to fit the expression and instead remain relatively small. For instance, $$f(x) = \pi(\frac{\sqrt{x}}{x-1})$$ comes out as

When we put (vertically) large expressions inside of parentheses (or brackets, or curly braces, etc.), the parentheses don't resize to fit the expression and instead remain relatively small. For instance, $$f(x) = \pi(\frac{\sqrt{x}}{x-1})$$ comes out as





\begin{figure}[!ht]
	\centering
	\begin{asy}
		import graph;
		size(300,0);
		int a=-1, b=1;
		real f(real x) {return x^3-x^2+2;}
		real g(real x) {return x^2;}
		draw(graph(f,a,b,operator ..),red);
		draw(graph(g,a,b,operator ..),0.5*orange);
		xaxis();
		int n=30;
		real width=(b-a)/(real) n;
		path w=graph(f,a,b,operator ..);
		path ww=graph(g,a,b,operator ..);
		path h=buildcycle((a,g(a))--(a,f(a)),w,(b,f(b))--(b,g(b)),ww);
		fill(h,0.5*orange);
		draw(h,0.5*yellow+linewidth(0.3mm));
		labelx("$a$",a);
		labelx("$b$",b);
		draw((a,0)--(a,g(a)),dotted);
		draw((b,0)--(b,g(b)),dotted);
		real m=a+0.73*(b-a);
		arrow("$f(x)$",(m,f(m)),N,red);
		arrow("$g(x)$",(m,g(m)),E,0.8cm,blue);
		int j=2;
		real xi=b-j*width;
		real xp=xi+width;
		real xm=0.5*(xi+xp);
		pair dot=(xm,0.5*(f(xm)+g(xm)));
		dot(dot,green+4.0);
		arrow("$\left(x,\frac{f(x)+g(x)}{2}\right)$",dot,NE,2cm,green);
	\end{asy}
	\caption{$f(x)$ wwww $\left(x,\frac{f(x)+g(x)}{2}\right)$ www}
\end{figure}



\begin{longtable}{>{\color{blue}}ccc>{\color{blue}}c>{\color{yellow}}cccccccccc}
	\caption{Combinaciones de los tres segmentos de la seccion aurea.}
	\label{tab:w1wwwww}                                                                         \\
	\toprule
	$Y_i$ & $f_i$ & $F_i$   & $F_i*$  & $h_i$  & $H_i$  & $H_i$  & $h_i\%$ & $H_i\%$ & $H_i*\%$ \\
	\midrule

	\endfirsthead
	\multicolumn{10}{c}{{\bfseries \tablename\ \thetable{} -- continua de la página anterior}}  \\
	\toprule
	$Y_i$ & $f_i$ & $F_i$   & $F_i*$  & $h_i$  & $H_i$  & $H_i$  & $h_i\%$ & $H_i\%$ & $H_i*\%$ \\
	\midrule
	\endhead
	\midrule
	\multicolumn{10}{c}{{Continúa en la proxima página}}                                        \\ \midrule
	\endfoot
	\bottomrule
	\endlastfoot
	2     & 1     & 1.0000  & 20.0000 & 0.0500 & 0.0500 & 0.0025 & 0.2500  & 0.2500  & 0.2500   \\
	3     & 2     & 3.0000  & 19.0000 & 0.1000 & 0.1500 & 0.0050 & 0.5000  & 0.5000  & 0.7500   \\
	4     & 5     & 8.0000  & 17.0000 & 0.2500 & 0.4000 & 0.0125 & 1.2500  & 1.2500  & 2.0000   \\
	5     & 7     & 15.0000 & 12.0000 & 0.3500 & 0.7500 & 0.0175 & 1.7500  & 1.7500  & 3.7500   \\
	6     & 4     & 19.0000 & 5.0000  & 0.2000 & 0.9500 & 0.0100 & 1.0000  & 1.0000  & 4.7500   \\
	7     & 1     & 20.0000 & 1.0000  & 0.0500 & 1.0000 & 0.0025 & 0.2500  & 0.2500  & 5.0000   \\
	      & 20    &         &         &        &        &        & 5       &         &          \\
\end{longtable}









\begin{longtable}{|c|cccccc|c|}
	\caption{www.}
	\label{tab:w}                                                                                                                     \\
	\hline
	Variables & $y_1$           & $y_2$           & $\ldots$ & $y_{j}$         & $\ldots$ & $y_n$    & Total                          \\
	\hline
	\endfirsthead
	\multicolumn{8}{c}{{\bfseries \tablename\ \thetable{} -- continua de la página anterior}}                                         \\
	\hline
	Variables & $y_1$           & $y_2$           & $\ldots$ & $y_{j}$         & $\ldots$ & $y_n$    & Total                          \\
	\hline
	\endhead
	\hline
	\multicolumn{8}{c}{{Continúa en la proxima página}}                                                                               \\
	\hline
	\endfoot
	\hline
	\endlastfoot
	$x_1$     & $f_{11}$        & $f_{12}$        & $\ldots$ & $f_{1j}$        & $\ldots$ & $f_{1e}$ & $f_1$                          \\
	$x_2$     & $f_{21}$        & $f_{22}$        & $\ldots$ & $f_{2j}$        & $\ldots$ & $f_{2e}$ & $f_2$                          \\
	$\ldots$  & $\ldots$        & $\ldots$        & $\ldots$ & $\ldots$        & $\ldots$ & $\ldots$ & $\ldots$                       \\
	$x_i$     & $f_{i1}$        & $f_{i2}$        & $\ldots$ & $f_{ij}$        & $\ldots$ & $f_{ie}$ & $f_j$                          \\
	$\ldots$  & $\ldots$        & $\ldots$        & $\ldots$ & $\ldots$        & $\ldots$ & $\ldots$ & $\ldots$                       \\
	$x_k$     & $f_{k1}$        & $f_{k2}$        & $\ldots$ & $f_{kj}$        & $\ldots$ & $f_{ke}$ & $f_k$                          \\
	\hline
	Total     & $f_{\dot \; 1}$ & $f_{\dot \; 2}$ & $\ldots$ & $f_{\dot \; j}$ & $\ldots$ & $f_n$    & $n=\sum_1^{n}\sum_1^{n}f_{ij}$ \\
\end{longtable}




%%%%%%%%%%%%%%%%%%%%%%%%%%%%%%%%%%%%%%%%222
%%%%%%%%%%%%%%%%%%%%%%%%%%%%%%%%%%%%%%%%
%%%%%%%%%%%%%%%%%%%%%%%%%%%%%%%%%%%%%%%%
%%%%%%%%%%%%%%%%%%%%%%%%%%%%%%%%%%%%%%%%
%%%%%%%%%%%%%%%%%%%%%%%%%%%%%%%%%%%%%%%%
%%%%%%%%%%%%%%%%%%%%%%%%%%%%%%%%%%%%%%%%
%%%%%%%%%%%%%%%%%%%%%%%%%%%%%%%%%%%%%%%%
%%%%%%%%%%%%%%%%%%%%%%%%%%%%%%%%%%%%%%%%
%%%%%%%%%%%%%%%%%%%%%%%%%%%%%%%%%%%%%%%%
%%%%%%%%%%%%%%%%%%%%%%%%%%%%%%%%%%%%%%%%
\part{Cálculo de probabilidades}
\chapter{Variables aleatorias}
\begin{defn}[Experimento aleatorio]
	El porceso en el cual ................... Se donta con $\epsilon$
\end{defn}

\begin{defn}[Espacio muestral]
	El porceso en el cual ................... Se donta con $\epsilon$
\end{defn}
\begin{defn}[Suceso]
	El porceso en el cual ................... Se donta con $\epsilon$
\end{defn}

\begin{defn}[Evento]
	El porceso en el cual ................... Se donta con $\epsilon$
\end{defn}



\begin{defn}[title]
	wwwwwwwwwwwwwwwwwwwwwwwwwwwwwww
\end{defn}

\begin{rem}[title]
	wwwwwwwwwwwwwwwwwwwww
\end{rem}




\begin{thm}[title]
	wwwwwwwwwwwwwwwwwwwwwwwwwwwwwwwwwwww
\end{thm}

\begin{prop}[title]
	wwwwwwwwwwwwww
\end{prop}
\chapter{www}



\begin{figure}[!ht]
	\centering
	\begin{asy}
		size(0,150);

		pen colour1=red;
		pen colour2=green;

		pair z0=(0,0);
		pair z1=(-1,0);
		pair z2=(1,0);
		real r=1.5;
		path c1=circle(z1,r);
		path c2=circle(z2,r);
		fill(c1,colour1);
		fill(c2,colour2);

		picture intersection;
		fill(intersection,c1,colour1+colour2);
		clip(intersection,c2);

		add(intersection);

		draw(c1);
		draw(c2);

		label("$A$",z1);
		label("$B$",z2);

		pair z=(0,-2);
		real m=3;
		margin BigMargin=Margin(0,m*dot(unit(z1-z),unit(z0-z)));

		draw(Label("$A\cap B$",0),conj(z)--z0,Arrow,BigMargin);
		draw(Label("$A\cup B$",0),z--z0,Arrow,BigMargin);
		draw(z--z1,Arrow,Margin(0,m));
		draw(z--z2,Arrow,Margin(0,m));

		shipout(bbox(0.25cm));

	\end{asy}
	\caption{}
\end{figure}

\chapter{Variables aleatorias discretas}

\chapter{Variables aleatorias continuas}


\begin{figure}[!ht]
	\centering
	\begin{asy}
		import graph;
		size(4inches,0);

		real f1(real x) {return (1+x^2);}
		real f2(real x) {return (4-x);}

		xaxis("$x$",LeftTicks,Arrow);
		yaxis("$y$",RightTicks,Arrow);

		draw("$y=1+x^2$",graph(f1,-2,1));
		dot((1,f1(1)),UnFill);

		draw("$y=4-x$",graph(f2,1,5),LeftSide,red,Arrow);
		dot((1,f2(1)),red);
	\end{asy}
	\caption{www2}
\end{figure}


\begin{figure}[!ht]
	\centering	\begin{asy}
		import graph;
		size(12cm,7cm,IgnoreAspect);
		typedef real realfcn(real);
		realfcn F(real p) {
				return new real(real t) {return 1/(sqrt(2*pi)*(1/(4*p)))*exp(-(t-1/(2*p))^2/(2*(1/(4*p))^2));};
			}
		for(int i=1; i < 7; ++i){
		real rho=(1/(4*i));
		real mu=(1/(2*i));
		draw(graph(F(i),-0.1,1, n=200, Hermite),Pen(i),
		"$\frac{1}{"+format( "%.3f", rho)+"\sqrt{2\pi}}{e^{-\frac{(x-"+format( "%.3f", mu)+")^2}{2("+format( "%.3f", rho)+")^2}}}$");
			}
			label("$\displaystyle\frac{1}{\rho\sqrt{2\pi}}{e^{-\frac{(x-\mu)^2}{2(\rho)^2}}}$",(0.5,8));
			xaxis("$x$",0.1,LeftTicks);
			yaxis("$y$",0,LeftTicks);
			//xaxis(BottomTop,LeftTicks);
			//yaxis(LeftRight,RightTicks(trailingzero));
			//yaxis("$y$",LeftRight,RightTicks(trailingzero));
		//attach(legend(),truepoint(E),20E,UnFill);
		attach(legend(2),(point(S).x,truepoint(S).y),1S);
	\end{asy}
	\caption{wwwwww}
\end{figure}


\begin{figure}[!ht]
	\centering	\begin{asy}
		import graph;
		size(12cm,7cm,IgnoreAspect);
		typedef real realfcn(real);
		realfcn F(real p) {
				return new real(real t) {return 1/(pi*(p)*(1+((t-5)/(p))^2));};
			}
		for(int i=1; i < 7; ++i){
				real rho=(1/(4*i));
				real mu=(1/(2*i));
				draw(graph(F(i),0,10, n=200, Hermite),Pen(i),
				"$\frac{1}{p\pi\left(1+\frac{t-5}{p^2}\right)}$");
			}
		label("$\displaystyle\frac{1}{p\pi\left(1+\frac{t-5}{p^2}\right)}$",(7,0.3));
		xaxis("$x$",0.1,LeftTicks);
		yaxis("$y$",0,LeftTicks);
		//xaxis(BottomTop,LeftTicks);
		//yaxis(LeftRight,RightTicks(trailingzero));
		//yaxis("$y$",LeftRight,RightTicks(trailingzero));
		//attach(legend(),truepoint(E),20E,UnFill);
		attach(legend(2),(point(S).x,truepoint(S).y),1S);
	\end{asy}
	\caption{wwwwww}
\end{figure}



























%%%%%%%%%%%%%%%%%%%%%%%%%%%%%%%%%%%%%%%%333
%%%%%%%%%%%%%%%%%%%%%%%%%%%%%%%%%%%%%%%%
%%%%%%%%%%%%%%%%%%%%%%%%%%%%%%%%%%%%%%%%
%%%%%%%%%%%%%%%%%%%%%%%%%%%%%%%%%%%%%%%%
%%%%%%%%%%%%%%%%%%%%%%%%%%%%%%%%%%%%%%%%
%%%%%%%%%%%%%%%%%%%%%%%%%%%%%%%%%%%%%%%%
%%%%%%%%%%%%%%%%%%%%%%%%%%%%%%%%%%%%%%%%
%%%%%%%%%%%%%%%%%%%%%%%%%%%%%%%%%%%%%%%%
%%%%%%%%%%%%%%%%%%%%%%%%%%%%%%%%%%%%%%%%
%%%%%%%%%%%%%%%%%%%%%%%%%%%%%%%%%%%%%%%%
\part{Inferencial estadística}
La inferencia estadística es primordialmente de naturaleza
inductiva y llega a generalizar respecto de las características de
una población valiéndose de observaciones empíricas de la
muestra.

Al utilizar estadísticas muestrales para estudiar un parámetro
de la población es muy normal que ambos sean diferentes y la
igualdad entre ambos sea mera coincidencia. La diferencia
entre la estadística muestral y el correspondiente parámetro
de la población se suele llamar error de estimación. Solo
conoceríamos dicho error si se conociera el parámetro
poblacional que por lo general se desconoce. La única forma
de tener alguna certeza al respecto es hacer todas las
observaciones posibles del total de la población; en la mayoría
de las aplicaciones prácticas es imposible o impracticable.


\chapter{Distribuciones muestrales}
La inferencia estadística es primordialmente de naturaleza
inductiva y llega a generalizar respecto de las características de
una población valiéndose de observaciones empíricas de la
muestra.

wwwwwwwwwwwwwwwwwwwwwwwwwwwwwwwwwwwwwwww


\begin{figure}[!ht]
	\centering
	\begin{asy}
		import graph;
		size(300,0);
		int a=0, b=2;
		real f(real x) {return 1/(sqrt(2*pi)*(0.5))*exp(-(x-1)^2/(2*(0.5)^2));}
		real g(real x) {return 0;}
		path w=graph(f,a,b,operator ..);
		draw(graph(f,a-1,b+1,operator ..),orange+linewidth(0.3mm));
		//draw(graph(g,a,b,operator ..),black);
		xaxis();
		int n=50;
		path h=(a,0)--w--(b,0)--cycle;
		fill(h,orange);
		draw(h,black+linewidth(0.3mm));
		labelx("$a$",a);
		labelx("$b$",b);
		pair mid=(a+0.5*(b-a),(f(a+0.5*(b-a))+g(a+0.5*(b-a)))/2);
		label("$90\%$",mid,white);
		real m=a+0.5*(b-a);
		real p=a-0.1;
		real q=b+0.1;
		//arrow("$f(x)$",(m,f(m)),N,red);
		arrow("$5\%$",(p,0.5*f(p)),NW,orange);
		dot((p,0.5*f(p)),orange);
		arrow("$5\%$",(q,0.5*f(q)),NE,orange);
		dot((q,0.5*f(q)),orange);
		//arrow("$g(x)$",(m,g(m)),dir(-90),0.8cm,blue);
	\end{asy}
	\caption{Normal}
\end{figure}



\cite{hilbert2020geometry}. ``\usebibentry{hilbert2020geometry}{title}''  this environment shares the counter of the previously defined thm environment.
\cite{reyes} \index{wwwww}\cite{www}


\index{www}
\index{www!wwww}
\index{www!wwww2}
\index{www!wwww3}
wwwwwwwwwwwwwwwwwwwwwwwwwwwwwwwwwwwwwww





%%%%%%%%%%%%%%%%%%%%%%%%%%%%%%%%%%%%%%%%
%%%%%%%%%%%%%%%%%%%%%%%%%%%%%%%%%%%%%%%%
%%%%%%%%%%%%%%%%%%%%%%%%%%%%%%%%%%%%%%%%
%%%%%%%%%%%%%%%%%%%%%%%%%%%%%%%%%%%%%%%%
%%%%%%%%%%%%%%%%%%%%%%%%%%%%%%%%%%%%%%%%
\bibliographystyle{apacite}
\bibliography{bb}
\addcontentsline{toc}{part}{Índices}
\printindex


%%%%%%%%%%%%%%%%%%%%%%%%%%%%%%%%%%%%%%%%
%%%%%%%%%%%%%%%%%%%%%%%%%%%%%%%%%%%%%%%%
%%%%%%%%%%%%%%%%%%%%%%%%%%%%%%%%%%%%%%%%
%%%%%%%%%%%%%%%%%%%%%%%%%%%%%%%%%%%%%%%%
%%%%%%%%%%%%%%%%%%%%%%%%%%%%%%%%%%%%%%%%
\appendix
\pagenumbering{roman}
\setcounter{page}{1}
\chapter{Sistemas de coordenadas}

\end{document}
/*
real[][] f(real a, real b, real c)
{
if (b^2-4*a*c<0) {
		real[][] w={
				{-b/2,(sqrt(abs(b^2-4*a*c)))/2*a,-b/2,(-b+sqrt(abs(b^2-4*a*c)))/2*a}//,{0,0}
			};
		return w;
	}
real[][] w={
{(-b+sqrt(abs(b^2-4*a*c)))/2*a,(-b-sqrt(abs(b^2-4*a*c)))/2*a}//,{0,0}
};
return w;
}

write(f(1,4,9));
write(f(1,4,9)[0][3]);
*/
